
{\actualityEN} \ldots

{\progressEN} \ldots

{\aimEN} \ldots

\ldots {\tasksEN}:
\begin{enumerate}
  \item \ldots
\end{enumerate}

{\researchObjectEN} \ldots

{\researchSubjectEN} \ldots

{\noveltyEN} \ldots

{\defpositionsEN} \ldots

{\methodsEN} \ldots

{\reliabilityEN} \ldots

{\influenceTheorEN} \ldots

{\influenceEN} \ldots

{\deploymentEN} \ldots

{\probationEN} \ldots

{\contributionEN} \ldots

%\publications\ Основные результаты по теме диссертации изложены в ХХ печатных изданиях~\cite{Sokolov,Gaidaenko,Lermontov,Management},
%Х из которых изданы в журналах, рекомендованных ВАК~\cite{Sokolov,Gaidaenko}, 
%ХХ --- в тезисах докладов~\cite{Lermontov,Management}.

\ifnumequal{\value{bibliosel}}{0}{% Встроенная реализация с загрузкой файла через движок bibtex8
%    \publications\ Основные результаты по теме диссертации изложены в XX печатных изданиях, 
%    X из которых изданы в журналах, рекомендованных ВАК, 
%    X "--- в тезисах докладов.%
}{% Реализация пакетом biblatex через движок biber
%Сделана отдельная секция, чтобы не отображались в списке цитированных материалов
\begin{refsection}[vakEN,papersEN,scopusEN,patentEN,programsEN]% Подсчет и нумерация авторских работ. Засчитываются только те, которые были прописаны внутри \nocite{}.
    %Чтобы сменить порядок разделов в сгрупированном списке литературы необходимо перетасовать следующие три строчки, а также команды в разделе \newcommand*{\insertbiblioauthorgrouped} в файле biblio/biblatex.tex
    \printbibliography[heading=countauthorscopusEN, env=countauthorscopusEN, keyword=biblioauthorscopusEN, section=3]%
    \printbibliography[heading=countauthorvakEN, env=countauthorvakEN, keyword=biblioauthorvakEN, section=3]%
    \printbibliography[heading=countauthorpatentEN, env=countauthorpatentEN, keyword=biblioauthorpatentEN, section=3]%
    \printbibliography[heading=countauthornotvakEN, env=countauthornotvakEN, keyword=biblioauthornotvakEN, section=3]%
    \printbibliography[heading=countauthorprogramsEN, env=countauthorprogramsEN, keyword=biblioauthorprogramsEN, section=3]%
    \printbibliography[heading=countauthorEN, env=countauthorEN, keyword=biblioauthorEN, section=3]%
    \nocite{%Порядок перечисления в этом блоке определяет порядок вывода в списке публикаций автора
        authorpaperscopus_1_EN,
        authorpaperscopus_2_EN,
        authorpapervak_1_EN,
        authorpatent_1_EN,
        authorprogram_1_EN,
        authorpaper_1_EN,
    }%
    {\publicationsEN}
    The primary results are presented in \arabic{citeauthorEN} publications;
    \arabic{citeauthorvakEN} of which are published in journals included into the List of the Higher Attestation Commission;
    \arabic{citeauthorscopusEN} are published in Scopus-indexed journals.
    The results described in the thesis are protected by one patent for invention.
    In addition, \arabic{citeauthorprogramsEN} computer programs have been registered.
\end{refsection}
\begin{refsection}[vakEN,papersEN,scopusEN,patentEN,programsEN]%Блок, позволяющий отобрать из всех работ автора наиболее значимые, и только их вывести в автореферате, но считать в блоке выше общее число работ
    \printbibliography[heading=countauthorscopusEN, env=countauthorscopusEN, keyword=biblioauthorscopusEN, section=4]%
    \printbibliography[heading=countauthorvakEN, env=countauthorvakEN, keyword=biblioauthorvakEN, section=4]%
    \printbibliography[heading=countauthorpatentEN, env=countauthorpatentEN, keyword=biblioauthorpatentEN, section=4]%
    \printbibliography[heading=countauthornotvakEN, env=countauthornotvakEN, keyword=biblioauthornotvakEN, section=4]%
    \printbibliography[heading=countauthorprogramsEN, env=countauthorprogramsEN, keyword=biblioauthorprogramsEN, section=4]%
    \printbibliography[heading=countauthorEN, env=countauthorEN, keyword=biblioauthorEN, section=4]%
    % Scopus
    \nocite{authorpaperscopus_1_EN}
    \nocite{authorpaperscopus_2_EN}
    \nocite{authorpapervak_1_EN}
    \nocite{authorpatent_1_EN}
    \nocite{authorprogram_1_EN}
    \nocite{authorpaper_1_EN}
\end{refsection}
}
