% Новые переменные, которые могут использоваться во всём проекте
% ГОСТ 7.0.11-2011
% 9.2 Оформление текста автореферата диссертации
% 9.2.1 Общая характеристика работы включает в себя следующие основные структурные
% элементы:
% актуальность темы исследования;
\newcommand{\actualityTXT}{Актуальность темы.}
% степень ее разработанности;
\newcommand{\progressTXT}{Степень разработанности темы.}
% цели и задачи;
\newcommand{\aimTXT}{Целью}
\newcommand{\tasksTXT}{задачи}
% научную новизну;
\newcommand{\noveltyTXT}{Научная новизна.}
% теоретическую и практическую значимость работы;
%\newcommand{\influenceTXT}{Теоретическая и практическая значимость}
% или чаще используют просто
\newcommand{\influenceTXT}{Практическая значимость работы.}
\newcommand{\influenceTheorTXT}{Теоретическая значимость работы}
% объект и предмет исследования
\newcommand{\researchObjectTXT}{Объект исследования}
\newcommand{\researchSubjectTXT}{Предмет исследования}
% методологию и методы исследования;
\newcommand{\methodsTXT}{Методология и методы исследования.}
% положения, выносимые на защиту;
\newcommand{\defpositionsTXT}{Основные положения, выносимые на~защиту:}
% степень достоверности и апробацию результатов.
\newcommand{\reliabilityTXT}{Достоверность}
\newcommand{\deploymentTXT}{Внедрение результатов работы.}
\newcommand{\probationTXT}{Апробация результатов работы.}

\newcommand{\contributionTXT}{Личный вклад автора}
\newcommand{\publicationsTXT}{Публикации.}
\newcommand{\thesisstructureTXT}{Объем и структура диссертации.}


\newcommand{\authorbibtitle}{Публикации автора по теме диссертации}
\newcommand{\vakbibtitle}{Статьи в изданиях из перечня ВАК}
\newcommand{\notvakbibtitle}{Статьи в прочих изданиях}
\newcommand{\scopusbibtitle}{Статьи в изданиях, индексируемых в базе цитирования Scopus}
\newcommand{\patentbibtitle}{Патенты}
\newcommand{\programsbibtitle}{Свидетельства о государственной регистрации программ для ЭВМ}
\newcommand{\fullbibtitle}{Список литературы} % (ГОСТ Р 7.0.11-2011, 4)
