%%% Основные сведения %%%
\newcommand{\thesisAuthorLastNameEN}{\todo{Secondname}}
\newcommand{\thesisAuthorOtherNamesEN}{\todo{Firstname Patronymic}}
\newcommand{\thesisAuthorInitialsEN}{\todo{F.\,P.}}
\newcommand{\thesisAuthorEN}             % Диссертация, ФИО автора
{%
    \texorpdfstring{% \texorpdfstring takes two arguments and uses the first for (La)TeX and the second for pdf
        \thesisAuthorLastNameEN~\thesisAuthorOtherNamesEN% так будет отображаться на титульном листе или в тексте, где будет использоваться переменная
    }{%
        \thesisAuthorLastNameEN, \thesisAuthorOtherNamesEN% эта запись для свойств pdf-файла. В таком виде, если pdf будет обработан программами для сбора библиографических сведений, будет правильно представлена фамилия.
    }
}
\newcommand{\thesisAuthorShortEN}        % Диссертация, ФИО автора с инициалами
{\thesisAuthorInitialsEN~\thesisAuthorLastNameEN}
\newcommand{\thesisTitleEN}              % Диссертация, название
{\todo{Thesis title}}

\newcommand{\thesisSpecialtyNumberEN}    % Диссертация, специальность, номер
{\thesisSpecialtyNumber}
\newcommand{\thesisSpecialtyTitleEN}     % Диссертация, специальность, название
{\todo{Speciality title}}
\newcommand{\thesisDegreeEN}             % Диссертация, ученая степень
{Doctor of Philosophy}
%\newcommand{\thesisDegreeShort}        % Диссертация, ученая степень, краткая запись
%{\todo{канд. физ.-мат. наук}}
\newcommand{\thesisCityEN}               % Диссертация, город написания диссертации
{Saint Petersburg}
\newcommand{\thesisYearEN}               % Диссертация, год написания диссертации
{\thesisYear}
\newcommand{\thesisOrganizationEN}       % Диссертация, организация
{Saint Petersburg National Research University of Information Technologies, Mechanics and Optics}
\newcommand{\thesisOrganizationShortEN}  % Диссертация, краткое название организации для доклада
{ITMO University}

\newcommand{\thesisInOrganizationEN}     % Диссертация, организация в предложном падеже: Работа выполнена в ...
{\thesisOrganizationEN}

\newcommand{\supervisorFioEN}            % Научный руководитель, ФИО
{\todo{Secondname Firstname Patronymic}}
\newcommand{\supervisorRegaliaEN}        % Научный руководитель, регалии
{\todo{Scientific Degree, Title}}
\newcommand{\supervisorFioShortEN}       % Научный руководитель, ФИО
{\todo{F.\,P.~Secondname}}
\newcommand{\supervisorRegaliaShortEN}   % Научный руководитель, регалии
{\todo{Sc. Deg.}}

\newcommand{\opponentOneFioEN}           % Оппонент 1, ФИО
{\todo{Secondname Firstname Patronymic}}
\newcommand{\opponentOneRegaliaEN}       % Оппонент 1, регалии
{\todo{Scientific Degree, Title}}
\newcommand{\opponentOneJobPlaceEN}      % Оппонент 1, место работы
{\todo{Job Place}}
\newcommand{\opponentOneJobPostEN}       % Оппонент 1, должность
{\todo{Position}}

\newcommand{\opponentTwoFioEN}           % Оппонент 2, ФИО
{\todo{Secondname Firstname Patronymic}}
\newcommand{\opponentTwoRegaliaEN}       % Оппонент 2, регалии
{\todo{Scientific Degree, Title}}
\newcommand{\opponentTwoJobPlaceEN}      % Оппонент 2, место работы
{\todo{Job Place}}
\newcommand{\opponentTwoJobPostEN}       % Оппонент 2, должность
{\todo{Position}}


\newcommand{\defenseDateEN}              % Защита, дата
{\todo{Mmmmmmmm~DDth, YYYY~at~XX~PM/AM}}
\newcommand{\defenseCouncilNumberEN}     % Защита, номер диссертационного совета
{\defenseCouncilNumber}
\newcommand{\defenseCouncilAddressEN}    % Защита, адрес учреждение диссертационного совета
{\todo{Address, room}}

\newcommand{\defenseSecretaryFioEN}      % Секретарь диссертационного совета, ФИО
{\todo{Secondname Firstname Patronymic}}
\newcommand{\defenseSecretaryRegaliaEN}  % Секретарь диссертационного совета, регалии
{\todo{Scientific Degree}}            % Для сокращений есть ГОСТы, например: ГОСТ Р 7.0.12-2011 + http://base.garant.ru/179724/#block_30000
