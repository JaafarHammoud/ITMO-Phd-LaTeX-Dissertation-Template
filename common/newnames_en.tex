% !TeX spellcheck = en_US

% Новые переменные, которые могут использоваться во всём проекте
% ГОСТ 7.0.11-2011
% 9.2 Оформление текста автореферата диссертации
% 9.2.1 Общая характеристика работы включает в себя следующие основные структурные
% элементы:
% актуальность темы исследования;
\newcommand{\actualityTXTEN}{Relevance.}
% степень ее разработанности;
\newcommand{\progressTXTEN}{State of the field.}
% цели и задачи;
\newcommand{\aimTXTEN}{The aim}
\newcommand{\tasksTXTEN}{tasks}
% научную новизну;
\newcommand{\noveltyTXTEN}{Scientific novelty.}
% теоретическую и практическую значимость работы;
% или чаще используют просто
\newcommand{\influenceTXTEN}{Practical significance.}
\newcommand{\influenceTheorTXTEN}{Theoretical significance}
% объект и предмет исследования
\newcommand{\researchObjectTXTEN}{Object of study}
\newcommand{\researchSubjectTXTEN}{Study subject}
% методологию и методы исследования;
\newcommand{\methodsTXTEN}{Research methodology and methods.}
% положения, выносимые на защиту;
\newcommand{\defpositionsTXTEN}{Principal statements of the thesis:}
% степень достоверности и апробацию результатов.
\newcommand{\reliabilityTXTEN}{The reliability}
\newcommand{\deploymentTXTEN}{Application.}
\newcommand{\probationTXTEN}{Approbation.}

\newcommand{\contributionTXTEN}{The personal contribution of the author}
\newcommand{\publicationsTXTEN}{Publications.}
\newcommand{\thesisstructureTXTEN}{Thesis structure.}


\newcommand{\authorbibtitleEN}{Publications of the author}
\newcommand{\vakbibtitleEN}{Articles in the journals included into the List of Higher Attestation Commission}
\newcommand{\notvakbibtitleEN}{Other articles}
\newcommand{\scopusbibtitleEN}{Articles in Scopus-indexed journals}
\newcommand{\patentbibtitleEN}{Patents}
\newcommand{\programsbibtitleEN}{Registered computer programs}
\newcommand{\fullbibtitleEN}{References} % (ГОСТ Р 7.0.11-2011, 4)
