%&preformat-disser
\RequirePackage[l2tabu,orthodox]{nag} % Раскомментировав, можно в логе получать рекомендации относительно правильного использования пакетов и предупреждения об устаревших и нерекомендуемых пакетах
% Формат А4, 14pt (ГОСТ Р 7.0.11-2011, 5.3.6)
\documentclass[a5paper,10pt,twoside,openany,article]{memoir}

\input{common/setup}            % общие настройки шаблона
\input{common/packages}         % Пакеты общие для диссертации и автореферата
\input{Dissertation/dispackages}    % Пакеты для диссертации
\input{Dissertation/userpackages}   % Пакеты для специфических пользовательских задач
\input{Synopsis/synpackages}  % Пакеты для автореферата
\input{Synopsis/userpackages} % Пакеты для специфических пользовательских задач


\input{Dissertation/setup}      % Упрощённые настройки шаблона
%\input{Synopsis/setup}        % Упрощённые настройки шаблона 

% Новые переменные, которые могут использоваться во всём проекте
% ГОСТ 7.0.11-2011
% 9.2 Оформление текста автореферата диссертации
% 9.2.1 Общая характеристика работы включает в себя следующие основные структурные
% элементы:
% актуальность темы исследования;
\newcommand{\actualityTXT}{Актуальность темы.}
% степень ее разработанности;
\newcommand{\progressTXT}{Степень разработанности темы.}
% цели и задачи;
\newcommand{\aimTXT}{Целью}
\newcommand{\tasksTXT}{задачи}
% научную новизну;
\newcommand{\noveltyTXT}{Научная новизна.}
% теоретическую и практическую значимость работы;
%\newcommand{\influenceTXT}{Теоретическая и практическая значимость}
% или чаще используют просто
\newcommand{\influenceTXT}{Практическая значимость работы.}
\newcommand{\influenceTheorTXT}{Теоретическая значимость работы}
% объект и предмет исследования
\newcommand{\researchObjectTXT}{Объект исследования}
\newcommand{\researchSubjectTXT}{Предмет исследования}
% методологию и методы исследования;
\newcommand{\methodsTXT}{Методология и методы исследования.}
% положения, выносимые на защиту;
\newcommand{\defpositionsTXT}{Основные положения, выносимые на~защиту:}
% степень достоверности и апробацию результатов.
\newcommand{\reliabilityTXT}{Достоверность}
\newcommand{\deploymentTXT}{Внедрение результатов работы.}
\newcommand{\probationTXT}{Апробация результатов работы.}

\newcommand{\contributionTXT}{Личный вклад автора}
\newcommand{\publicationsTXT}{Публикации.}
\newcommand{\thesisstructureTXT}{Объем и структура диссертации.}


\newcommand{\authorbibtitle}{Публикации автора по теме диссертации}
\newcommand{\vakbibtitle}{Статьи в изданиях из перечня ВАК}
\newcommand{\notvakbibtitle}{Статьи в прочих изданиях}
\newcommand{\scopusbibtitle}{Статьи в изданиях, индексируемых в базе цитирования Scopus}
\newcommand{\patentbibtitle}{Патенты}
\newcommand{\programsbibtitle}{Свидетельства о государственной регистрации программ для ЭВМ}
\newcommand{\fullbibtitle}{Список литературы} % (ГОСТ Р 7.0.11-2011, 4)
         % Новые переменные, для всего проекта
% !TeX spellcheck = en_US

% Новые переменные, которые могут использоваться во всём проекте
% ГОСТ 7.0.11-2011
% 9.2 Оформление текста автореферата диссертации
% 9.2.1 Общая характеристика работы включает в себя следующие основные структурные
% элементы:
% актуальность темы исследования;
\newcommand{\actualityTXTEN}{Relevance.}
% степень ее разработанности;
\newcommand{\progressTXTEN}{State of the field.}
% цели и задачи;
\newcommand{\aimTXTEN}{The aim}
\newcommand{\tasksTXTEN}{tasks}
% научную новизну;
\newcommand{\noveltyTXTEN}{Scientific novelty.}
% теоретическую и практическую значимость работы;
% или чаще используют просто
\newcommand{\influenceTXTEN}{Practical significance.}
\newcommand{\influenceTheorTXTEN}{Theoretical significance}
% объект и предмет исследования
\newcommand{\researchObjectTXTEN}{Object of study}
\newcommand{\researchSubjectTXTEN}{Study subject}
% методологию и методы исследования;
\newcommand{\methodsTXTEN}{Research methodology and methods.}
% положения, выносимые на защиту;
\newcommand{\defpositionsTXTEN}{Principal statements of the thesis:}
% степень достоверности и апробацию результатов.
\newcommand{\reliabilityTXTEN}{The reliability}
\newcommand{\deploymentTXTEN}{Application.}
\newcommand{\probationTXTEN}{Approbation.}

\newcommand{\contributionTXTEN}{The personal contribution of the author}
\newcommand{\publicationsTXTEN}{Publications.}
\newcommand{\thesisstructureTXTEN}{Thesis structure.}


\newcommand{\authorbibtitleEN}{Publications of the author}
\newcommand{\vakbibtitleEN}{Articles in the journals included into the List of Higher Attestation Commission}
\newcommand{\notvakbibtitleEN}{Other articles}
\newcommand{\scopusbibtitleEN}{Articles in Scopus-indexed journals}
\newcommand{\patentbibtitleEN}{Patents}
\newcommand{\programsbibtitleEN}{Registered computer programs}
\newcommand{\fullbibtitleEN}{References} % (ГОСТ Р 7.0.11-2011, 4)
         % Новые переменные, для всего проекта

%%% Основные сведения %%%
\newcommand{\thesisAuthorLastName}{\todo{Фамилия}}
\newcommand{\thesisAuthorOtherNames}{\todo{Имя Отчество}}
\newcommand{\thesisAuthorInitials}{\todo{И.\,О.}}
\newcommand{\thesisAuthor}             % Диссертация, ФИО автора
{%
    \texorpdfstring{% \texorpdfstring takes two arguments and uses the first for (La)TeX and the second for pdf
        \thesisAuthorLastName~\thesisAuthorOtherNames% так будет отображаться на титульном листе или в тексте, где будет использоваться переменная
    }{%
        \thesisAuthorLastName, \thesisAuthorOtherNames% эта запись для свойств pdf-файла. В таком виде, если pdf будет обработан программами для сбора библиографических сведений, будет правильно представлена фамилия.
    }
}
\newcommand{\thesisAuthorShort}        % Диссертация, ФИО автора с инициалами
{\thesisAuthorInitials~\thesisAuthorLastName}
\newcommand{\thesisTitle}              % Диссертация, название
{\todo{Название диссертационной работы}}
\newcommand{\thesisSpecialtyNumber}    % Диссертация, специальность, номер
{\todo{XX.XX.XX}}
\newcommand{\thesisSpecialtyTitle}     % Диссертация, специальность, название
{\todo{Название специальности}}
\newcommand{\thesisDegree}             % Диссертация, ученая степень
{\todo{кандидата физико-математических наук}}
\newcommand{\thesisDegreeShort}        % Диссертация, ученая степень, краткая запись
{\todo{канд. физ.-мат. наук}}
\newcommand{\thesisCity}               % Диссертация, город написания диссертации
{Санкт-Петербург}
\newcommand{\thesisYear}               % Диссертация, год написания диссертации
{\todo{20XX}}
\newcommand{\thesisOrganization}       % Диссертация, организация
{Санкт-Петербургский национальный исследовательский университет информационных технологий, механики и оптики}
\newcommand{\thesisOrganizationShort}  % Диссертация, краткое название организации для доклада
{Университет ИТМО}

\newcommand{\thesisInOrganization}     % Диссертация, организация в предложном падеже: Работа выполнена в ...
{Санкт-Петербургском национальном исследовательском университете информационных технологий, механики и оптики}

\newcommand{\supervisorFio}            % Научный руководитель, ФИО
{\todo{Фамилия Имя Отчество}}
\newcommand{\supervisorRegalia}        % Научный руководитель, регалии
{\todo{уч. степень, уч. звание}}
\newcommand{\supervisorFioShort}       % Научный руководитель, ФИО
{\todo{И.\,О.~Фамилия}}
\newcommand{\supervisorRegaliaShort}   % Научный руководитель, регалии
{\todo{уч.~ст.,~уч.~зв.}}


\newcommand{\opponentOneFio}           % Оппонент 1, ФИО
{\todo{Фамилия Имя Отчество}}
\newcommand{\opponentOneRegalia}       % Оппонент 1, регалии
{\todo{доктор физико-математических наук, профессор}}
\newcommand{\opponentOneJobPlace}      % Оппонент 1, место работы
{\todo{Не очень длинное название для места работы}}
\newcommand{\opponentOneJobPost}       % Оппонент 1, должность
{\todo{старший научный сотрудник}}

\newcommand{\opponentTwoFio}           % Оппонент 2, ФИО
{\todo{Фамилия Имя Отчество}}
\newcommand{\opponentTwoRegalia}       % Оппонент 2, регалии
{\todo{кандидат физико-математических наук}}
\newcommand{\opponentTwoJobPlace}      % Оппонент 2, место работы
{\todo{Основное место работы c длинным длинным длинным длинным названием}}
\newcommand{\opponentTwoJobPost}       % Оппонент 2, должность
{\todo{старший научный сотрудник}}


\newcommand{\defenseDate}              % Защита, дата
{\todo{DD mmmmmmmm YYYY~г.~в~XX часов}}
\newcommand{\defenseCouncilNumber}     % Защита, номер диссертационного совета
{\todo{123.456.78}}
\newcommand{\defenseCouncilAddress}    % Защита, адрес учреждение диссертационного совета
{\todo{Адрес, аудитория}}

\newcommand{\defenseSecretaryFio}      % Секретарь диссертационного совета, ФИО
{\todo{Фамилия Имя Отчество}}
\newcommand{\defenseSecretaryRegalia}  % Секретарь диссертационного совета, регалии
{\todo{д-р~физ.-мат. наук}}            % Для сокращений есть ГОСТы, например: ГОСТ Р 7.0.12-2011 + http://base.garant.ru/179724/#block_30000

% To avoid conflict with beamer class use \providecommand
\providecommand{\keywords}%            % Ключевые слова для метаданных PDF диссертации и автореферата
{}
             % Основные сведения
%%% Основные сведения %%%
\newcommand{\thesisAuthorLastNameEN}{\todo{Secondname}}
\newcommand{\thesisAuthorOtherNamesEN}{\todo{Firstname Patronymic}}
\newcommand{\thesisAuthorInitialsEN}{\todo{F.\,P.}}
\newcommand{\thesisAuthorEN}             % Диссертация, ФИО автора
{%
    \texorpdfstring{% \texorpdfstring takes two arguments and uses the first for (La)TeX and the second for pdf
        \thesisAuthorLastNameEN~\thesisAuthorOtherNamesEN% так будет отображаться на титульном листе или в тексте, где будет использоваться переменная
    }{%
        \thesisAuthorLastNameEN, \thesisAuthorOtherNamesEN% эта запись для свойств pdf-файла. В таком виде, если pdf будет обработан программами для сбора библиографических сведений, будет правильно представлена фамилия.
    }
}
\newcommand{\thesisAuthorShortEN}        % Диссертация, ФИО автора с инициалами
{\thesisAuthorInitialsEN~\thesisAuthorLastNameEN}
\newcommand{\thesisTitleEN}              % Диссертация, название
{\todo{Thesis title}}






\newcommand{\thesisSpecialtyNumberEN}    % Диссертация, специальность, номер
{\thesisSpecialtyNumber}
\newcommand{\thesisSpecialtyTitleEN}     % Диссертация, специальность, название
{\todo{Speciality title}}
\newcommand{\thesisDegreeEN}             % Диссертация, ученая степень
{Doctor of Philosophy}
%\newcommand{\thesisDegreeShort}        % Диссертация, ученая степень, краткая запись
%{\todo{канд. физ.-мат. наук}}
\newcommand{\thesisCityEN}               % Диссертация, город написания диссертации
{Saint Petersburg}
\newcommand{\thesisYearEN}               % Диссертация, год написания диссертации
{\thesisYear}
\newcommand{\thesisOrganizationEN}       % Диссертация, организация
{Saint Petersburg National Research University of Information Technologies, Mechanics and Optics}
\newcommand{\thesisOrganizationShortEN}  % Диссертация, краткое название организации для доклада
{ITMO University}

\newcommand{\thesisInOrganizationEN}     % Диссертация, организация в предложном падеже: Работа выполнена в ...
{\thesisOrganizationEN}

\newcommand{\supervisorFioEN}            % Научный руководитель, ФИО
{\todo{Secondname Firstname Patronymic}}
\newcommand{\supervisorRegaliaEN}        % Научный руководитель, регалии
{\todo{Scientific Degree, Title}}
\newcommand{\supervisorFioShortEN}       % Научный руководитель, ФИО
{\todo{F.\,P.~Secondname}}
\newcommand{\supervisorRegaliaShortEN}   % Научный руководитель, регалии
{\todo{Sc. Deg.}}








\newcommand{\opponentOneFioEN}           % Оппонент 1, ФИО
{\todo{Secondname Firstname Patronymic}}
\newcommand{\opponentOneRegaliaEN}       % Оппонент 1, регалии
{\todo{Scientific Degree, Title}}
\newcommand{\opponentOneJobPlaceEN}      % Оппонент 1, место работы
{\todo{Job Place}}
\newcommand{\opponentOneJobPostEN}       % Оппонент 1, должность
{\todo{Position}}

\newcommand{\opponentTwoFioEN}           % Оппонент 2, ФИО
{\todo{Secondname Firstname Patronymic}}
\newcommand{\opponentTwoRegaliaEN}       % Оппонент 2, регалии
{\todo{Scientific Degree, Title}}
\newcommand{\opponentTwoJobPlaceEN}      % Оппонент 2, место работы
{\todo{Job Place}}
\newcommand{\opponentTwoJobPostEN}       % Оппонент 2, должность
{\todo{Position}}


\newcommand{\defenseDateEN}              % Защита, дата
{\todo{DD mmmmmmmm YYYY~year~at~XX PM/AM}}
\newcommand{\defenseCouncilNumberEN}     % Защита, номер диссертационного совета
{\defenseCouncilNumber}
\newcommand{\defenseCouncilAddressEN}    % Защита, адрес учреждение диссертационного совета
{\todo{Address, room}}

\newcommand{\defenseSecretaryFioEN}      % Секретарь диссертационного совета, ФИО
{\todo{Secondname Firstname Patronymic}}
\newcommand{\defenseSecretaryRegaliaEN}  % Секретарь диссертационного совета, регалии
{\todo{Scientific Degree}}            % Для сокращений есть ГОСТы, например: ГОСТ Р 7.0.12-2011 + http://base.garant.ru/179724/#block_30000
          % Основные сведения, на английском языке
\input{common/fonts}            % Определение шрифтов (частичное)
\input{common/styles}           % Стили общие для диссертации и автореферата
\input{Dissertation/disstyles}  % Стили для диссертации
\input{Dissertation/userstyles} % Стили для специфических пользовательских задач
\input{Synopsis/synstyles}    % Стили для автореферата
\input{Synopsis/userstyles}   % Стили для специфических пользовательских задач

%%% Библиография. Выбор движка для реализации %%%
\ifnumequal{\value{bibliosel}}{0}{%
    \input{biblio/predefined}   % Встроенная реализация с загрузкой файла через движок bibtex8
}{
    \input{biblio/biblatex}     % Реализация пакетом biblatex через движок biber
}

%%% Управление компиляцией отдельных частей диссертации %%%
% Необходимо сначала иметь полностью скомпилированный документ, чтобы все
% промежуточные файлы были в наличии
% Затем, для вывода отдельных частей можно воспользоваться командой \includeonly
% Ниже примеры использования команды:
%
%\includeonly{Dissertation/part2}
%\includeonly{Dissertation/contents,Dissertation/appendix,Dissertation/conclusion}
%
% Если все команды закомментированы, то документ будет выведен в PDF файл полностью

\begin{document}

\input{common/renames}                 % Переопределение именований

%%% Структура диссертации (ГОСТ Р 7.0.11-2011, 4)
\include{Dissertation/title}           % Титульный лист
% Титульный лист (ГОСТ Р 7.0.11-2001, 5.1)
\thispagestyle{empty}
\begin{center}
\thesisOrganizationEN
\end{center}


\vspace{0pt plus2fill} %число перед fill = кратность относительно некоторого расстояния fill, кусками которого заполнены пустые места

\noindent
\begin{tabu} to\linewidth {X[l,p]X[r,p]}
    \includegraphics[height=5.5cm]{logo_en}
    &
    \multirow[p]{1}{*}[1.2cm]{As a manuscript} \\
\end{tabu}


\vspace{0pt plus6fill} %число перед fill = кратность относительно некоторого расстояния fill, кусками которого заполнены пустые места
\begin{center}%
    {\large \thesisAuthorEN}
\end{center}%
%
\vspace{0pt plus1fill} %число перед fill = кратность относительно некоторого расстояния fill, кусками которого заполнены пустые места
\begin{center}%
    \textbf {\large %\MakeUppercase
        \thesisTitleEN}
    
    \vspace{0pt plus2fill} %число перед fill = кратность относительно некоторого расстояния fill, кусками которого заполнены пустые места
    {%\small
        Specialty \thesisSpecialtyNumberEN~---
        
        <<\thesisSpecialtyTitleEN>>
    }
    
    \vspace{0pt plus2fill} %число перед fill = кратность относительно некоторого расстояния fill, кусками которого заполнены пустые места
    A thesis submitted in fulfillment of the requirements \par for the degree of 
    \thesisDegreeEN
\end{center}%
%
\vspace{0pt plus4fill} %число перед fill = кратность относительно некоторого расстояния fill, кусками которого заполнены пустые места
\begin{flushright}%
    Scientific adviser:
    
    \supervisorRegaliaEN
    
    \supervisorFioEN
\end{flushright}%
%
\vspace{0pt plus4fill} %число перед fill = кратность относительно некоторого расстояния fill, кусками которого заполнены пустые места
\begin{center}%
    {\thesisCityEN~--- \thesisYearEN}
\end{center}%        % Титульный лист на английском языке

% Реферат на русском языке
\setcounter{page}{2}  % TODO С какого номера страницы правильно начать?
\renewcommand{\thepage}{I-\arabic{page}}

\include{Synopsis/content}

% Реферат на английском языке
\setcounter{page}{2}  % TODO С какого номера страницы правильно начать?
\renewcommand{\thepage}{II-\arabic{page}}

\thispagestyle{empty}

\begin{center}%
    \textbf{\Large Synopsis}
\end{center}%

\vspace{0.008\paperheight plus0.2fill}

\noindent The research was carried out at {\thesisInOrganizationEN}.

\vspace{0.008\paperheight plus1fill}
\noindent%
\begin{tabularx}{\textwidth}{@{}lX@{}}
    Scientific adviser:   & \supervisorRegaliaEN\par
    \textbf{\supervisorFioEN}
    \vspace{0.013\paperheight}\\
    Official opponents:  &
        \textbf{\opponentOneFioEN,}\par
        \opponentOneRegaliaEN,\par
        \opponentOneJobPlaceEN,\par
        \opponentOneJobPostEN\par
        \vspace{0.01\paperheight}
        \textbf{\opponentTwoFioEN}, \par
        \opponentTwoRegaliaEN,\par
        \opponentTwoJobPlaceEN,\par
        \opponentTwoJobPostEN
\end{tabularx}
\vspace{0.008\paperheight plus1fill}


\noindent The defense will be held on \defenseDateEN~at the meeting of the ITMO University Dissertation Council \defenseCouncilNumberEN~at \defenseCouncilAddressEN.


\vspace{0.008\paperheight plus0.1fill}

\noindent The thesis is available in the Library of Saint Petersburg National Research University of Information Technologies, Mechanics and Optics, 49 Kronversky pr., Saint Petersburg, Russia and on fppo.ifmo.ru website.

\vspace{0.008\paperheight plus1fill}
\noindent%
\begin{tabularx}{\textwidth}{@{}%
        >{\raggedright\arraybackslash}b{12em}@{}
        >{\centering\arraybackslash}X
        r
        @{}}
    Science Secretary of the\par
    ITMO University\par
    Dissertation Council \defenseCouncilNumberEN,\par
    \defenseSecretaryRegaliaEN
    &
    \IfFileExists{images/secretary-signature.png}{\includegraphics[width=2cm]{secretary-signature.png}}
    &
    \defenseSecretaryFioEN
\end{tabularx} 

\newpage

\section*{Thesis overview}

\newcommand{\actualityEN}{\underline{\textbf{\actualityTXTEN}}}
\newcommand{\progressEN}{\underline{\textbf{\progressTXTEN}}}
\newcommand{\aimEN}{\underline{{\textbf\aimTXTEN}}}
\newcommand{\tasksEN}{\underline{\textbf{\tasksTXTEN}}}
\newcommand{\noveltyEN}{\underline{\textbf{\noveltyTXTEN}}}
\newcommand{\influenceEN}{\underline{\textbf{\influenceTXTEN}}}
\newcommand{\influenceTheorEN}{\underline{\textbf{\influenceTheorTXTEN}}}
\newcommand{\methodsEN}{\underline{\textbf{\methodsTXTEN}}}
\newcommand{\defpositionsEN}{\underline{\textbf{\defpositionsTXTEN}}}
\newcommand{\reliabilityEN}{\underline{\textbf{\reliabilityTXTEN}}}
\newcommand{\probationEN}{\underline{\textbf{\probationTXTEN}}}
\newcommand{\contributionEN}{\underline{\textbf{\contributionTXTEN}}}
\newcommand{\publicationsEN}{\underline{\textbf{\publicationsTXTEN}}}
\newcommand{\researchObjectEN}{\underline{\textbf{\researchObjectTXTEN}}}
\newcommand{\researchSubjectEN}{\underline{\textbf{\researchSubjectTXTEN}}}
\newcommand{\deploymentEN}{\underline{\textbf{\deploymentTXTEN}}}
\newcommand{\thesisstructureEN}{\underline{\textbf{\thesisstructureTXTEN}}}

% TODO
%
{\actualityEN} \ldots

{\progressEN} \ldots

{\aimEN} \ldots

\ldots {\tasksEN}:
\begin{enumerate}
  \item \ldots
\end{enumerate}

{\researchObjectEN} \ldots

{\researchSubjectEN} \ldots

{\noveltyEN} \ldots

{\defpositionsEN} \ldots

{\methodsEN} \ldots

{\reliabilityEN} \ldots

{\influenceTheorEN} \ldots

{\influenceEN} \ldots

{\deploymentEN} \ldots

{\probationEN} \ldots

{\contributionEN} \ldots

%\publications\ Основные результаты по теме диссертации изложены в ХХ печатных изданиях~\cite{Sokolov,Gaidaenko,Lermontov,Management},
%Х из которых изданы в журналах, рекомендованных ВАК~\cite{Sokolov,Gaidaenko}, 
%ХХ --- в тезисах докладов~\cite{Lermontov,Management}.

\ifnumequal{\value{bibliosel}}{0}{% Встроенная реализация с загрузкой файла через движок bibtex8
%    \publications\ Основные результаты по теме диссертации изложены в XX печатных изданиях, 
%    X из которых изданы в журналах, рекомендованных ВАК, 
%    X "--- в тезисах докладов.%
}{% Реализация пакетом biblatex через движок biber
%Сделана отдельная секция, чтобы не отображались в списке цитированных материалов
\begin{refsection}[vakEN,papersEN,scopusEN,patentEN,programsEN]% Подсчет и нумерация авторских работ. Засчитываются только те, которые были прописаны внутри \nocite{}.
    %Чтобы сменить порядок разделов в сгрупированном списке литературы необходимо перетасовать следующие три строчки, а также команды в разделе \newcommand*{\insertbiblioauthorgrouped} в файле biblio/biblatex.tex
    \printbibliography[heading=countauthorscopusEN, env=countauthorscopusEN, keyword=biblioauthorscopusEN, section=3]%
    \printbibliography[heading=countauthorvakEN, env=countauthorvakEN, keyword=biblioauthorvakEN, section=3]%
    \printbibliography[heading=countauthorpatentEN, env=countauthorpatentEN, keyword=biblioauthorpatentEN, section=3]%
    \printbibliography[heading=countauthornotvakEN, env=countauthornotvakEN, keyword=biblioauthornotvakEN, section=3]%
    \printbibliography[heading=countauthorprogramsEN, env=countauthorprogramsEN, keyword=biblioauthorprogramsEN, section=3]%
    \printbibliography[heading=countauthorEN, env=countauthorEN, keyword=biblioauthorEN, section=3]%
    \nocite{%Порядок перечисления в этом блоке определяет порядок вывода в списке публикаций автора
        authorpaperscopus_1_EN,
        authorpaperscopus_2_EN,
        authorpapervak_1_EN,
        authorpatent_1_EN,
        authorprogram_1_EN,
        authorpaper_1_EN,
    }%
    {\publicationsEN}
    % Здесь нужно привести текст в соответствие с числительными
    The primary results are presented in \arabic{citeauthorEN} publications;
    \arabic{citeauthorvakEN} of which are published in journals included into the List of the Higher Attestation Commission;
    \arabic{citeauthorscopusEN} are published in Scopus-indexed journals.
    The results described in the thesis are protected by
    \arabic{citeauthorpatentEN}
    patent for invention.
    In addition, \arabic{citeauthorprogramsEN} computer programs have been registered.
\end{refsection}
\begin{refsection}[vakEN,papersEN,scopusEN,patentEN,programsEN]%Блок, позволяющий отобрать из всех работ автора наиболее значимые, и только их вывести в автореферате, но считать в блоке выше общее число работ
    \printbibliography[heading=countauthorscopusEN, env=countauthorscopusEN, keyword=biblioauthorscopusEN, section=4]%
    \printbibliography[heading=countauthorvakEN, env=countauthorvakEN, keyword=biblioauthorvakEN, section=4]%
    \printbibliography[heading=countauthorpatentEN, env=countauthorpatentEN, keyword=biblioauthorpatentEN, section=4]%
    \printbibliography[heading=countauthornotvakEN, env=countauthornotvakEN, keyword=biblioauthornotvakEN, section=4]%
    \printbibliography[heading=countauthorprogramsEN, env=countauthorprogramsEN, keyword=biblioauthorprogramsEN, section=4]%
    \printbibliography[heading=countauthorEN, env=countauthorEN, keyword=biblioauthorEN, section=4]%
    % Scopus
    \nocite{authorpaperscopus_1_EN}
    \nocite{authorpaperscopus_2_EN}
    \nocite{authorpapervak_1_EN}
    \nocite{authorpatent_1_EN}
    \nocite{authorprogram_1_EN}
    \nocite{authorpaper_1_EN}
\end{refsection}
}
 % Характеристика работы по структуре во введении и в автореферате не отличается (ГОСТ Р 7.0.11, пункты 5.3.1 и 9.2.1), потому её загружаем из одного и того же внешнего файла, предварительно задав форму выделения некоторым параметрам

%Диссертационная работа была выполнена при поддержке грантов ...

%\underline{\textbf{Объем и структура работы.}} Диссертация состоит из~введения,
%четырех глав, заключения и~приложения. Полный объем диссертации
%\textbf{ХХХ}~страниц текста с~\textbf{ХХ}~рисунками и~5~таблицами. Список
%литературы содержит \textbf{ХХX}~наименование.

\section*{Thesis contents}
% TODO
%\input{common/concl_en}


% TODO In English!!!
\ifdefmacro{\microtypesetup}{\microtypesetup{protrusion=false}}{} % не рекомендуется применять пакет микротипографики к автоматически генерируемому списку литературы
\ifnumequal{\value{bibliosel}}{0}{% Встроенная реализация с загрузкой файла через движок bibtex8
  \renewcommand{\bibname}{\large \authorbibtitle}
  \nocite{*}
  \insertbiblioauthorgrouped           % Подключаем Bib-базы
  %\insertbiblioother   % !!! bibtex не умеет работать с несколькими библиографиями !!!
}{% Реализация пакетом biblatex через движок biber
  \ifnumgreater{\value{usefootcite}}{0}{
%  \nocite{*} % Невидимая цитата всех работ, позволит вывести все работы автора
  \insertbiblioauthorcited      % Вывод процитированных в автореферате работ автора
  }{
  \insertbiblioauthorgrouped           % Вывод всех работ автора
%  \insertbiblioauthorgrouped    % Вывод всех работ автора, сгруппированных по источникам
%  \insertbiblioauthorimportant  % Вывод наиболее значимых работ автора (определяется в файле characteristic во второй section)
  %\insertbiblioother            % Вывод списка литературы, на которую ссылались в тексте автореферата
  }
}
\ifdefmacro{\microtypesetup}{\microtypesetup{protrusion=true}}{}


% Основной текст диссертации
\setcounter{page}{1}
\renewcommand{\thepage}{\arabic{page}}

%\setcounter{usefootcite}{0}

% Чтобы оглавление основной части диссертации начиналось с чётной ("левой") страницы
\ifoddpage
    \null
\else
    \newpage
    \thispagestyle{empty}
    \null
    \newpage
\fi
\include{Dissertation/contents}        % Оглавление
\include{Dissertation/introduction}    % Введение
\include{Dissertation/part1}           % Глава 1
\include{Dissertation/part2}           % Глава 2
\include{Dissertation/part3}           % Глава 3
\include{Dissertation/conclusion}      % Заключение
\include{Dissertation/acronyms}        % Список сокращений и условных обозначений
\include{Dissertation/dictionary}      % Словарь терминов
\include{Dissertation/references}      % Список литературы
\include{Dissertation/lists}           % Списки таблиц и изображений (иллюстративный материал)

%%% Настройки для приложений
\appendix
% Оформление заголовков приложений ближе к ГОСТ:
\setlength{\midchapskip}{20pt}
\renewcommand*{\afterchapternum}{\par\nobreak\vskip \midchapskip}
\renewcommand\thechapter{\Asbuk{chapter}} % Чтобы приложения русскими буквами нумеровались

\include{Dissertation/appendix}        % Приложения

\end{document}
