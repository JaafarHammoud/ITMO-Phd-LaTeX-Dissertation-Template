\thispagestyle{empty}

\begin{center}%
    \textbf{\Large Synopsis}
\end{center}%

\vspace{0.008\paperheight plus0.2fill}

\noindent The research was carried out at {\thesisInOrganizationEN}.

\vspace{0.008\paperheight plus1fill}
\noindent%
\begin{tabularx}{\textwidth}{@{}lX@{}}
    Scientific adviser:   & \supervisorRegaliaEN\par
    \textbf{\supervisorFioEN}
    \vspace{0.013\paperheight}\\
    Official opponents:  &
        \textbf{\opponentOneFioEN,}\par
        \opponentOneRegaliaEN,\par
        \opponentOneJobPlaceEN,\par
        \opponentOneJobPostEN\par
        \vspace{0.01\paperheight}
        \textbf{\opponentTwoFioEN}, \par
        \opponentTwoRegaliaEN,\par
        \opponentTwoJobPlaceEN,\par
        \opponentTwoJobPostEN
\end{tabularx}
\vspace{0.008\paperheight plus1fill}


\noindent The defense will be held on \defenseDateEN~at the meeting of the ITMO University Dissertation Council \defenseCouncilNumberEN~at \defenseCouncilAddressEN.


\vspace{0.008\paperheight plus0.1fill}

\noindent The thesis is available in the Library of Saint Petersburg National Research University of Information Technologies, Mechanics and Optics, 49 Kronversky pr., Saint Petersburg, Russia and on fppo.ifmo.ru website.

\vspace{0.008\paperheight plus1fill}
\noindent%
\begin{tabularx}{\textwidth}{@{}%
        >{\raggedright\arraybackslash}b{12em}@{}
        >{\centering\arraybackslash}X
        r
        @{}}
    Science Secretary of the\par
    ITMO University\par
    Dissertation Council \defenseCouncilNumberEN,\par
    \defenseSecretaryRegaliaEN
    &
    \IfFileExists{images/secretary-signature.png}{\includegraphics[width=2cm]{secretary-signature.png}}
    &
    \defenseSecretaryFioEN
\end{tabularx} 

\newpage

\section*{Thesis overview}

\newcommand{\actualityEN}{\underline{\textbf{\actualityTXTEN}}}
\newcommand{\progressEN}{\underline{\textbf{\progressTXTEN}}}
\newcommand{\aimEN}{\underline{{\textbf\aimTXTEN}}}
\newcommand{\tasksEN}{\underline{\textbf{\tasksTXTEN}}}
\newcommand{\noveltyEN}{\underline{\textbf{\noveltyTXTEN}}}
\newcommand{\influenceEN}{\underline{\textbf{\influenceTXTEN}}}
\newcommand{\influenceTheorEN}{\underline{\textbf{\influenceTheorTXTEN}}}
\newcommand{\methodsEN}{\underline{\textbf{\methodsTXTEN}}}
\newcommand{\defpositionsEN}{\underline{\textbf{\defpositionsTXTEN}}}
\newcommand{\reliabilityEN}{\underline{\textbf{\reliabilityTXTEN}}}
\newcommand{\probationEN}{\underline{\textbf{\probationTXTEN}}}
\newcommand{\contributionEN}{\underline{\textbf{\contributionTXTEN}}}
\newcommand{\publicationsEN}{\underline{\textbf{\publicationsTXTEN}}}
\newcommand{\researchObjectEN}{\underline{\textbf{\researchObjectTXTEN}}}
\newcommand{\researchSubjectEN}{\underline{\textbf{\researchSubjectTXTEN}}}
\newcommand{\deploymentEN}{\underline{\textbf{\deploymentTXTEN}}}
\newcommand{\thesisstructureEN}{\underline{\textbf{\thesisstructureTXTEN}}}


{\actualityEN} \ldots

{\progressEN} \ldots

{\aimEN} \ldots

\ldots {\tasksEN}:
\begin{enumerate}
  \item \ldots
\end{enumerate}

{\researchObjectEN} \ldots

{\researchSubjectEN} \ldots

{\noveltyEN} \ldots

{\defpositionsEN} \ldots

{\methodsEN} \ldots

{\reliabilityEN} \ldots

{\influenceTheorEN} \ldots

{\influenceEN} \ldots

{\deploymentEN} \ldots

{\probationEN} \ldots

{\contributionEN} \ldots

%\publications\ Основные результаты по теме диссертации изложены в ХХ печатных изданиях~\cite{Sokolov,Gaidaenko,Lermontov,Management},
%Х из которых изданы в журналах, рекомендованных ВАК~\cite{Sokolov,Gaidaenko}, 
%ХХ --- в тезисах докладов~\cite{Lermontov,Management}.

\ifnumequal{\value{bibliosel}}{0}{% Встроенная реализация с загрузкой файла через движок bibtex8
%    \publications\ Основные результаты по теме диссертации изложены в XX печатных изданиях, 
%    X из которых изданы в журналах, рекомендованных ВАК, 
%    X "--- в тезисах докладов.%
}{% Реализация пакетом biblatex через движок biber
%Сделана отдельная секция, чтобы не отображались в списке цитированных материалов
\begin{refsection}[vakEN,papersEN,scopusEN,patentEN,programsEN]% Подсчет и нумерация авторских работ. Засчитываются только те, которые были прописаны внутри \nocite{}.
    %Чтобы сменить порядок разделов в сгрупированном списке литературы необходимо перетасовать следующие три строчки, а также команды в разделе \newcommand*{\insertbiblioauthorgrouped} в файле biblio/biblatex.tex
    \printbibliography[heading=countauthorscopusEN, env=countauthorscopusEN, keyword=biblioauthorscopusEN, section=3]%
    \printbibliography[heading=countauthorvakEN, env=countauthorvakEN, keyword=biblioauthorvakEN, section=3]%
    \printbibliography[heading=countauthorpatentEN, env=countauthorpatentEN, keyword=biblioauthorpatentEN, section=3]%
    \printbibliography[heading=countauthornotvakEN, env=countauthornotvakEN, keyword=biblioauthornotvakEN, section=3]%
    \printbibliography[heading=countauthorprogramsEN, env=countauthorprogramsEN, keyword=biblioauthorprogramsEN, section=3]%
    \printbibliography[heading=countauthorEN, env=countauthorEN, keyword=biblioauthorEN, section=3]%
    \nocite{%Порядок перечисления в этом блоке определяет порядок вывода в списке публикаций автора
        authorpaperscopus_1_EN,
        authorpaperscopus_2_EN,
        authorpapervak_1_EN,
        authorpatent_1_EN,
        authorprogram_1_EN,
        authorpaper_1_EN,
    }%
    {\publicationsEN}
    % Здесь нужно привести текст в соответствие с числительными
    The primary results are presented in \arabic{citeauthorEN} publications;
    \arabic{citeauthorvakEN} of which are published in journals included into the List of the Higher Attestation Commission;
    \arabic{citeauthorscopusEN} are published in Scopus-indexed journals.
    The results described in the thesis are protected by
    \arabic{citeauthorpatentEN}
    patent for invention.
    In addition, \arabic{citeauthorprogramsEN} computer programs have been registered.
\end{refsection}
\begin{refsection}[vakEN,papersEN,scopusEN,patentEN,programsEN]%Блок, позволяющий отобрать из всех работ автора наиболее значимые, и только их вывести в автореферате, но считать в блоке выше общее число работ
    \printbibliography[heading=countauthorscopusEN, env=countauthorscopusEN, keyword=biblioauthorscopusEN, section=4]%
    \printbibliography[heading=countauthorvakEN, env=countauthorvakEN, keyword=biblioauthorvakEN, section=4]%
    \printbibliography[heading=countauthorpatentEN, env=countauthorpatentEN, keyword=biblioauthorpatentEN, section=4]%
    \printbibliography[heading=countauthornotvakEN, env=countauthornotvakEN, keyword=biblioauthornotvakEN, section=4]%
    \printbibliography[heading=countauthorprogramsEN, env=countauthorprogramsEN, keyword=biblioauthorprogramsEN, section=4]%
    \printbibliography[heading=countauthorEN, env=countauthorEN, keyword=biblioauthorEN, section=4]%
    % Scopus
    \nocite{authorpaperscopus_1_EN}
    \nocite{authorpaperscopus_2_EN}
    \nocite{authorpapervak_1_EN}
    \nocite{authorpatent_1_EN}
    \nocite{authorprogram_1_EN}
    \nocite{authorpaper_1_EN}
\end{refsection}
}
 % Характеристика работы по структуре во введении и в автореферате не отличается (ГОСТ Р 7.0.11, пункты 5.3.1 и 9.2.1), потому её загружаем из одного и того же внешнего файла, предварительно задав форму выделения некоторым параметрам

%Диссертационная работа была выполнена при поддержке грантов ...

%\underline{\textbf{Объем и структура работы.}} Диссертация состоит из~введения,
%четырех глав, заключения и~приложения. Полный объем диссертации
%\textbf{ХХХ}~страниц текста с~\textbf{ХХ}~рисунками и~5~таблицами. Список
%литературы содержит \textbf{ХХX}~наименование.

\section*{Thesis contents}
Contents, in English.

\textbf{Conclusion}.

%% Согласно ГОСТ Р 7.0.11-2011:
%% 5.3.3 В заключении диссертации излагают итоги выполненного исследования, рекомендации, перспективы дальнейшей разработки темы.
%% 9.2.3 В заключении автореферата диссертации излагают итоги данного исследования, рекомендации и перспективы дальнейшей разработки темы.
Conclusions, in English.



\ifdefmacro{\microtypesetup}{\microtypesetup{protrusion=false}}{} % не рекомендуется применять пакет микротипографики к автоматически генерируемому списку литературы
\ifnumequal{\value{bibliosel}}{0}{% Встроенная реализация с загрузкой файла через движок bibtex8
  \renewcommand{\bibname}{\large \authorbibtitleEN}
  \nocite{*}
  \insertbiblioauthorgroupedEN           % Подключаем Bib-базы
  %\insertbiblioother   % !!! bibtex не умеет работать с несколькими библиографиями !!!
}{% Реализация пакетом biblatex через движок biber
  \ifnumgreater{\value{usefootcite}}{0}{
%  \nocite{*} % Невидимая цитата всех работ, позволит вывести все работы автора
  \insertbiblioauthorcitedEN      % Вывод процитированных в автореферате работ автора
  }{
  \insertbiblioauthorgroupedEN           % Вывод всех работ автора
%  \insertbiblioauthorgrouped    % Вывод всех работ автора, сгруппированных по источникам
%  \insertbiblioauthorimportant  % Вывод наиболее значимых работ автора (определяется в файле characteristic во второй section)
  %\insertbiblioother            % Вывод списка литературы, на которую ссылались в тексте автореферата
  }
}
\ifdefmacro{\microtypesetup}{\microtypesetup{protrusion=true}}{}
